% Options for packages loaded elsewhere
\PassOptionsToPackage{unicode}{hyperref}
\PassOptionsToPackage{hyphens}{url}
\PassOptionsToPackage{dvipsnames,svgnames,x11names}{xcolor}
%
\documentclass[
  letterpaper,
  DIV=11,
  numbers=noendperiod]{scrartcl}

\usepackage{amsmath,amssymb}
\usepackage{iftex}
\ifPDFTeX
  \usepackage[T1]{fontenc}
  \usepackage[utf8]{inputenc}
  \usepackage{textcomp} % provide euro and other symbols
\else % if luatex or xetex
  \usepackage{unicode-math}
  \defaultfontfeatures{Scale=MatchLowercase}
  \defaultfontfeatures[\rmfamily]{Ligatures=TeX,Scale=1}
\fi
\usepackage{lmodern}
\ifPDFTeX\else  
    % xetex/luatex font selection
\fi
% Use upquote if available, for straight quotes in verbatim environments
\IfFileExists{upquote.sty}{\usepackage{upquote}}{}
\IfFileExists{microtype.sty}{% use microtype if available
  \usepackage[]{microtype}
  \UseMicrotypeSet[protrusion]{basicmath} % disable protrusion for tt fonts
}{}
\makeatletter
\@ifundefined{KOMAClassName}{% if non-KOMA class
  \IfFileExists{parskip.sty}{%
    \usepackage{parskip}
  }{% else
    \setlength{\parindent}{0pt}
    \setlength{\parskip}{6pt plus 2pt minus 1pt}}
}{% if KOMA class
  \KOMAoptions{parskip=half}}
\makeatother
\usepackage{xcolor}
\setlength{\emergencystretch}{3em} % prevent overfull lines
\setcounter{secnumdepth}{-\maxdimen} % remove section numbering
% Make \paragraph and \subparagraph free-standing
\makeatletter
\ifx\paragraph\undefined\else
  \let\oldparagraph\paragraph
  \renewcommand{\paragraph}{
    \@ifstar
      \xxxParagraphStar
      \xxxParagraphNoStar
  }
  \newcommand{\xxxParagraphStar}[1]{\oldparagraph*{#1}\mbox{}}
  \newcommand{\xxxParagraphNoStar}[1]{\oldparagraph{#1}\mbox{}}
\fi
\ifx\subparagraph\undefined\else
  \let\oldsubparagraph\subparagraph
  \renewcommand{\subparagraph}{
    \@ifstar
      \xxxSubParagraphStar
      \xxxSubParagraphNoStar
  }
  \newcommand{\xxxSubParagraphStar}[1]{\oldsubparagraph*{#1}\mbox{}}
  \newcommand{\xxxSubParagraphNoStar}[1]{\oldsubparagraph{#1}\mbox{}}
\fi
\makeatother


\providecommand{\tightlist}{%
  \setlength{\itemsep}{0pt}\setlength{\parskip}{0pt}}\usepackage{longtable,booktabs,array}
\usepackage{calc} % for calculating minipage widths
% Correct order of tables after \paragraph or \subparagraph
\usepackage{etoolbox}
\makeatletter
\patchcmd\longtable{\par}{\if@noskipsec\mbox{}\fi\par}{}{}
\makeatother
% Allow footnotes in longtable head/foot
\IfFileExists{footnotehyper.sty}{\usepackage{footnotehyper}}{\usepackage{footnote}}
\makesavenoteenv{longtable}
\usepackage{graphicx}
\makeatletter
\newsavebox\pandoc@box
\newcommand*\pandocbounded[1]{% scales image to fit in text height/width
  \sbox\pandoc@box{#1}%
  \Gscale@div\@tempa{\textheight}{\dimexpr\ht\pandoc@box+\dp\pandoc@box\relax}%
  \Gscale@div\@tempb{\linewidth}{\wd\pandoc@box}%
  \ifdim\@tempb\p@<\@tempa\p@\let\@tempa\@tempb\fi% select the smaller of both
  \ifdim\@tempa\p@<\p@\scalebox{\@tempa}{\usebox\pandoc@box}%
  \else\usebox{\pandoc@box}%
  \fi%
}
% Set default figure placement to htbp
\def\fps@figure{htbp}
\makeatother

\KOMAoption{captions}{tableheading}
\makeatletter
\@ifpackageloaded{caption}{}{\usepackage{caption}}
\AtBeginDocument{%
\ifdefined\contentsname
  \renewcommand*\contentsname{Índice}
\else
  \newcommand\contentsname{Índice}
\fi
\ifdefined\listfigurename
  \renewcommand*\listfigurename{Lista de Figuras}
\else
  \newcommand\listfigurename{Lista de Figuras}
\fi
\ifdefined\listtablename
  \renewcommand*\listtablename{Lista de Tabelas}
\else
  \newcommand\listtablename{Lista de Tabelas}
\fi
\ifdefined\figurename
  \renewcommand*\figurename{Figura}
\else
  \newcommand\figurename{Figura}
\fi
\ifdefined\tablename
  \renewcommand*\tablename{Tabela}
\else
  \newcommand\tablename{Tabela}
\fi
}
\@ifpackageloaded{float}{}{\usepackage{float}}
\floatstyle{ruled}
\@ifundefined{c@chapter}{\newfloat{codelisting}{h}{lop}}{\newfloat{codelisting}{h}{lop}[chapter]}
\floatname{codelisting}{Listagem}
\newcommand*\listoflistings{\listof{codelisting}{Lista de Listagens}}
\makeatother
\makeatletter
\makeatother
\makeatletter
\@ifpackageloaded{caption}{}{\usepackage{caption}}
\@ifpackageloaded{subcaption}{}{\usepackage{subcaption}}
\makeatother

\ifLuaTeX
\usepackage[bidi=basic]{babel}
\else
\usepackage[bidi=default]{babel}
\fi
\babelprovide[main,import]{brazilian}
% get rid of language-specific shorthands (see #6817):
\let\LanguageShortHands\languageshorthands
\def\languageshorthands#1{}
\usepackage{bookmark}

\IfFileExists{xurl.sty}{\usepackage{xurl}}{} % add URL line breaks if available
\urlstyle{same} % disable monospaced font for URLs
\hypersetup{
  pdftitle={introR: introdução à linguagem R},
  pdflang={pt-BR},
  colorlinks=true,
  linkcolor={blue},
  filecolor={Maroon},
  citecolor={Blue},
  urlcolor={Blue},
  pdfcreator={LaTeX via pandoc}}


\title{introR: introdução à linguagem R}
\author{}
\date{}

\begin{document}
\maketitle


\subsubsection{Docentes}\label{docentes}

Prof.~Dr.~Maurício Humberto Vancine

Prof.~Dr.~Mario Moura (responsável)

\subsubsection{Carga horária}\label{carga-horuxe1ria}

30 h (2 créditos)

\subsubsection{Participantes}\label{participantes}

10 alunos (+ 5 especiais)

\subsubsection{Datas e horários}\label{datas-e-horuxe1rios}

Teórico-prático: 14/10/2025 a 17/10/2025 (9-17h)

\subsubsection{Repositório da
disciplina}\label{reposituxf3rio-da-disciplina}

\url{https://github.com/course-intror}

\subsubsection{Resumo}\label{resumo}

A disciplina tem como foco principal o controle de versão com git/GitHub
e a introdução à programação em R, aplicada a dados ecológicos. São
abordados tanto os aspectos teóricos quanto práticos do controle de
versão, incluindo o uso do software git e de repositórios remotos no
GitHub. Além disso, são apresentados os principais tópicos de
programação em R, desde recursos do Base R até o tidyverse, com ênfase
no manejo e na visualização de dados ecológicos, bem como em tópicos
avançados de programação na linguagem. Serão abordados os seguintes
temas: (1) controle de versão com git e GitHub, (2) introdução à
programação em R (Base R), (3) introdução à programação em R
(tidyverse), (4) tópicos avançados em programação no R. A carga horária
total será de 30 horas, onde nos três dias iniciais serão ministrados 20
horas de aulas teórico-práticas. As 10 horas restantes serão
direcionadas à formulação e execução de um projeto com dados reais, como
forma de avaliação para compor a nota final da disciplina. Ao final da
disciplina, os alunos devem ser capazes de utilizar git/GitHub para
trabalho colaborativo em ciência e aplicar fundamentos e técnicas
avançadas de programação em R para manejo, análise e visualização de
dados ecológicos.

\subsubsection{Conteúdo}\label{conteuxfado}

\subsubsection{1 Controle de versão com git e
GitHub}\label{controle-de-versuxe3o-com-git-e-github}

\begin{enumerate}
\def\labelenumi{\arabic{enumi}.}
\tightlist
\item
  Conferindo os computadores
\item
  Controle de versão
\item
  git e GitHub
\item
  Detalhes do GitHub
\item
  Criando um repositório
\item
  Configuração: git config
\item
  Controle de versão na prática
\item
  Iniciando localmente: git init
\item
  Iniciando remotamente: fork e git clone
\item
  Versionamento: git status, git add e git commit
\item
  Ignorando: .gitignore
\item
  Histórico: git log e git show
\item
  Diferença: git diff
\item
  Desfazer: git restore, git revert e git reset
\item
  Ramificações: git branch, git switch e git merge
\item
  Remoto: git remote, git push e git pull
\item
  GitHub: Pull request
\item
  Conflitos
\item
  Interface gráfica do RStudio
\end{enumerate}

\subsubsection{2 introdução à programação em R (Base
R)}\label{introduuxe7uxe3o-uxe0-programauxe7uxe3o-em-r-base-r}

\begin{enumerate}
\def\labelenumi{\arabic{enumi}.}
\tightlist
\item
  Linguagem R
\item
  RStudio
\item
  Console
\item
  Scripts
\item
  Operadores
\item
  Objetos
\item
  Funções
\item
  Pacotes
\item
  Ajuda
\item
  Ambiente
\item
  Citações
\item
  Principais erros
\item
  Atributos dos objetos
\item
  Manipulação de dados unidimensionais
\item
  Manipulação de dados multidimensionais
\item
  Valores faltantes e especiais
\item
  Diretório de trabalho
\item
  Importar dados
\item
  Conferência de dados importados
\item
  Exportar dados
\end{enumerate}

\subsubsection{3 Introdução à programação em R
(tidyverse)}\label{introduuxe7uxe3o-uxe0-programauxe7uxe3o-em-r-tidyverse}

\begin{enumerate}
\def\labelenumi{\arabic{enumi}.}
\tightlist
\item
  Contextualização
\item
  tidyverse
\item
  here
\item
  readr, readxl e writexl
\item
  tibble
\item
  magrittr (pipe - \%\textgreater\%)
\item
  tidyr
\item
  dplyr
\item
  stringr
\item
  forcats
\item
  lubridate
\item
  purrr
\item
  Pacotes para produção de gráficos
\item
  Gramática dos gráficos
\item
  Principal material de estudo
\item
  Principais tipos de gráficos
\item
  Histograma e Densidade
\item
  Gráfico de setores
\item
  Gráfico de barras
\item
  Gráfico de caixas
\item
  Gráfico de dispersão
\item
  Gráfico pareado
\item
  Combinando gráficos
\item
  Gráficos animados
\item
  Gráficos interativos
\item
  Gráficos usando interface
\end{enumerate}

\subsubsection{4 tópicos avançados em programação no
R}\label{tuxf3picos-avanuxe7ados-em-programauxe7uxe3o-no-r}

\begin{enumerate}
\def\labelenumi{\arabic{enumi}.}
\tightlist
\item
  Tabelas de frequência
\item
  Frequência absoluta e relativa
\item
  Função: table
\item
  Medidas de posição e dispersão
\item
  Funções: apply, lapply, sapply, tapply
\item
  Controle de fluxo
\item
  Condicional: if, else e else if
\item
  Estruturas de repetição
\item
  Laços: for, while e repeat
\item
  Comandos: break e next
\item
  Funções
\item
  Funções externas: source
\end{enumerate}

\subsubsection{Referências}\label{referuxeancias}

Chang W. 2013. R Graphics Cookbook: Practical Recipes for Visualizing
Data. 2 ed.~O'Reilly Media. \url{https://r-graphics.org}

Chacon S., Straub B. 2014. Pro Git. 2 ed.~Apress.
\url{https://git-scm.com/book/en/v2}

Cotton R. 2013. Learning R: A Step-by-Step Function Guide to Data
Analysis. O'Reilly Media.

Davies TM. 2016. The Book of R: A First Course in Programming and
Statistics. No Starch Press.

Damiani A, Milz B, Lente C, Falbel D, Correa F, Trecenti J, Luduvice N,
Lacerda T, Amorim W. 2025. Ciência de Dados em R.
\href{https://livro.curso-r.com/}{https://livro.curso-r.com}

Da Silva FR, Gonçalves-Souza T, Paterno GB, Provete DB, Vancine MH.
2022. Análises Ecológicas no R. Recife: Nupeea. Bauru, SP: Canal 6.
\url{https://analises-ecologicas.com}

Engel C. 2019. Introduction to R. \url{https://cengel.github.io/R-intro}

Hastle T, Tibshirani R, Friedman J. 2016. The Elements of Statistical
Learning: Data Mining, Inference, and Prediction. 2 ed.~Springer.
\url{https://web.stanford.edu/~hastie/ElemStatLearn} Healy K. 2019. Data
Visualization: a practical introduction. Princeton University Press.

James G, Witten D, Hastie T, Tibshirani R. 2013. An Introduction to
Statistical Learning: with Applications in R. 2.ed. Springer.
\url{http://faculty.marshall.usc.edu/gareth-james/ISL}

Kabacoff RI. 2015. R in Action: Data analysis and graphics with R. 2.ed.
Manning.

Lander JP. 2017. R for Everyone: Advanced Analytics and Graphics.
Addison-Wesley Professional.

Matloff N. 2011. The Art of R Programming: A Tour of Statistical
Software Design. No Starch Press.

Oliveira PF, Guerra S, Mcdonnell, R. 2018. Ciência de dados com R --
Introdução. IBPAD. \url{https://cdr.ibpad.com.br}

R Core Team. 2020. R: A language and environment for statistical
computing. R Foundation for Statistical Computing, Vienna, Austria.
\url{https://www.r-project.org}

Teetor P. 2011. R Cookbook. O'Reilly Media.
\url{http://www.cookbook-r.com}

Wickham, H., Cetinkaya-Rundel, M., Grolemund, G. 2023. R for Data
Science: Import, Tidy, Transform, Visualize, and Model Data. O'Reilly
Media. \url{https://r4ds.hadley.nz}

Wickham H. 2019. Advanced R. 2 ed.~Chapman and Hall/CRC.
\url{https://adv-r.hadley.nz}

Wickham H. 2020. ggplot2: Elegant Graphics for Data Analysis. 3
ed.~Springer. \url{https://ggplot2-book.org}

Wilk CO. 2019. Fundamentals of Data Visualization: A Primer on Making
Informative and Compelling Figures. O'Reilly Media.
\url{https://serialmentor.com/dataviz}




\end{document}
